\documentclass[handout,pdf]{beamer}
\usepackage[utf8]{inputenc}


\usefonttheme{professionalfonts}
\usepackage{amsmath}
\usepackage{amssymb}
\usepackage{mathtools}
\usepackage{dirtytalk}
\usepackage{graphicx} 
\usepackage{tikz}
\usepackage{tikz-cd}
\newcommand{\longsquiggly}{\xymatrix{{}\ar@{~>}[r]&{}}}

\usepackage{mathptmx}

\usepackage{thmtools}
%Enumerar
\usepackage{enumerate}

%solves align issue with pause
\makeatletter
\let\save@measuring@true\measuring@true
\def\measuring@true{%
  \save@measuring@true
  \def\beamer@sortzero##1{\beamer@ifnextcharospec{\beamer@sortzeroread{##1}}{}}%
  \def\beamer@sortzeroread##1<##2>{}%
  \def\beamer@finalnospec{}%
}


\usepackage{pst-node}


%Commands 
\newcommand{\im}{\mathrm{im}\,}
\newcommand{\dif}{\mathrm{d}}
\usetikzlibrary{calc, intersections}
\newcommand{\modul}{\; \mathrm{mod}\;}
\newcommand{\Hom}{\mathrm{Hom}}
\newcommand{\acal}{\mathcal{A}}
\newcommand{\divergence}{\mathrm{div} \,}
\newcommand{\bcal}{\mathcal{B}}
\newcommand{\coker}{\mathrm{coker}\,}
\newcommand{\pfrak}{\mathfrak{p}}
\newcommand{\spec}{\mathrm{Spec}\,}
\newcommand{\fcal}{\mathscr{F}}
\newcommand{\ocal}{\mathscr{O}}
\newcommand{\hcal}{\mathscr{H}}
\newcommand{\tcal}{\mathcal{T}}
\newcommand{\gcal}{\mathscr{G}}
\newcommand{\cl}{\mathrm{cl}}
\newcommand{\ucal}{\mathcal{U}}
\newcommand{\mcal}{\mathcal{M}}
\newcommand{\ical}{\mathcal{I}}
\newcommand{\res}{\mathrm{res}}
\newcommand{\continuous}{\mathrm{C}}
\newcommand{\ccal}{\mathcal{C}}
\newcommand{\obj}{\mathrm{Obj}}
\newcommand{\mor}{\mathrm{Mor}}
\newcommand{\oposite}{\ccal^{\mathrm{op}}}
\newcommand{\dcal}{\mathcal{D}}
\newcommand{\sets}{\mathbf{Set}}
\newcommand{\cat}{\mathbf{Cat}}
\newcommand{\ecal}{\mathcal{E}}
\newcommand{\jcal}{\mathscr{J}}
\newcommand{\group}{\mathbf{Group}}
\newcommand{\wcal}{\mathcal{W}}
\newcommand{\ring}{\mathbf{Ring}}
\newcommand{\topo}{\mathbf{Top}}
\newcommand{\vecto}{\mathbf{Vect}}
\newcommand{\mfrak}{\mathfrak{m}}
\newcommand{\ide}{\mathrm{id}}
\newcommand{\fun}{\mathbf{Fun}}
\newcommand{\vcal}{\mathcal{V}}
\newcommand{\pcal}{\mathcal{P}}
\newcommand{\lcal}{\mathscr{L}}
\newcommand{\colim}{\mathop{\mathrm{colim}}\limits}
\newcommand{\nerve}{\mathrm{N}}
\newcommand{\spt}{\mathrm{spt \, }}
\newcommand{\conj}{\overline{z}}

\newcommand{\derived}{\mathrm{D}}
\newcommand{\rcal}{\mathcal{R}}
\newcommand{\kcal}{\mathcal{K}}
\newcommand{\homcal}{\textit{Hom}}
\newcommand{\ext}{\mathrm{Ext}}
%\DeclareMathOperator{\lan}{Lan}
%\DeclareMathOperator{\ran}{Ran}
%\DeclareMathOperator{\tor}{Tor}
%\DeclareMathOperator{\kom}{Kom}
%\DeclareMathOperator{\pder}{PDer}
%\DeclareMathOperator{\der}{Der}
\newcommand{\cone}{\mathsf{C}}

\usepackage[mathcal]{eucal}

%indice
\usepackage{makeidx}
\makeindex


%number of frames
\setbeamertemplate{footline}[frame number]{}

%Fuente
\usepackage{mathptmx}

\usepackage{csquotes}

%Diagramas
\usepackage[all]{xy}
\usepackage{tikz}
\usetikzlibrary{patterns, positioning, arrows}

%Teoremas, Definiciones y demás
\theoremstyle{plain}
\newtheorem{thm}{Teorema}[section]
\newtheorem{cor}[thm]{Corolario}
\newtheorem{lema}[thm]{Lema}
\newtheorem{prop}[thm]{Proposición}
\newtheorem*{thm*}{Teorema}


\theoremstyle{definition}
\newtheorem{defi}[thm]{Definición}
\newtheorem{ej}[thm]{Ejemplo}
\newtheorem*{not*}{Notación}

\usefonttheme[onlymath]{serif}

\theoremstyle{remark}
\newtheorem*{rem}{Observación}
\newtheorem*{ej*}{Ejemplo}
\newtheorem*{defi*}{Definición}


%Tema
\usetheme{Darmstadt}
\usecolortheme{beaver}

\title{Rational BV-Algebra In String Topology}
\author[Álvaro Jiménez]{Álvaro Jiménez Calvo}
\date{}

%Secciones resaltadas e índice al principio de cada sección
\AtBeginSection[]
{
  \begin{frame}
    \frametitle{Índice}
    \tableofcontents[currentsection]
  \end{frame}
}

\setbeamertemplate{section in toc}[square]

\setbeamerfont{section number projected}{size=\large}
\setbeamercolor{section number projected}{bg=red!60!black,fg=white}

\begin{document}

\begin{frame}
\titlepage
\end{frame}

\begin{frame}
\frametitle{Plan for the Talk}
\tableofcontents[hideallsubsections]
\end{frame}


\section[]{Goals}

\begin{frame}{Main Theorem}
The goal of this talk is to prove the following statement: \pause 
\begin{alertblock}{Theorem}
If $M$ is $1$-connected and the field of coefficitnets has characteristic zero, then \pause 
\begin{itemize}
\setbeamercolor{item projected}{bg=red!60!black,fg=white}
  \setbeamertemplate{enumerate items}[square]
    \item there is a BV-structure on $HH^{\bullet}(C^{\bullet}(M), C^{\bullet}(M))$ extending the structure of a Gerstenhaber algebra, \pause 
    \item there is an isomorphism of BV-algebras 
    $$\mathbb{H}_{\bullet}(\text{L}M) \cong HH^{\bullet}(C^{\bullet}(M), C^{\bullet}(M)) \, .$$
\end{itemize}
\end{alertblock}
\end{frame}




\begin{frame}{Notation I}
All manifolds are assumed to be smooth orientable, closed, with $\dim(M) = m$. \\\pause 
\vspace{0.75cm}

Denote by 
$$\text{L}M = \text{Map}(\mathbb{S}^1, M)$$
the \textit{loop space of $M$}. \pause Its \textit{shifted homology} will be denoted by 
$$\mathbb{H}_{\bullet}(\text{L}M) = H_{\bullet + m}(\text{L}M) \, .$$
\vspace{0.5cm}

$C_{\bullet}(M)$ and $C^{\bullet}(M)$ denote the singular chains and cochains of $M$. \pause 
\vspace{0.5cm}

Convention: \pause complexes will have cohomological degree $V^i = V_{-i}$. 
\end{frame}

\vspace{0.5cm}


\begin{frame}{Notation II}
$A$ will denote a DG-algebra, \pause $s A$ denotes the \say{shift by $1$}, i.e. 
$$(sA)^i = A^{i+1} \, ,$$
\pause and $T(A)$ denotes the \textit{tensor coalgebra} 
$$T(A) = \bigoplus_{i = 0}(sA)^{\otimes i}\, .$$
\end{frame}

\section[]{Motivation}

\begin{frame}{Motivation}

Assume the coefficients ring is a field. \pause Jones \cite{jones} showed there is a linear isomorphism
$$HH_{\bullet}(C^{\bullet}(M), C^{\bullet}(M)) \cong H^{\bullet}(\text{L}M) \, .$$\pause

\vspace{0.5cm}
By duality, $H_{\bullet}(\text{L}M) \cong HH^{\bullet}(C^{\bullet}(M) , C_{\bullet}(M))$. \pause 

\vspace{0.5cm}

Next, the cap product induces an isomorphism of graded vector spaces \pause 
$$HH^{\bullet}(C^{\bullet}(M), C_{\bullet}(M)) \cong HH^{\bullet - m}(C^{\bullet}(M), C^{\bullet}(M)) \, ,$$
\pause and therefore an isomorphism 
$$\mathbb{H}_{\bullet}(\text{L}M) \cong HH^{\bullet}(C^{\bullet}(M), C^{\bullet}(M)) \, .$$

\end{frame}

\begin{frame}{Motivation (II)}
On the other hand, Lambrechts and Stanley (\cite{stanley}) showed the following: \pause 
\begin{alertblock}{Theorem}
There exists a commutative differential graded algebra $A$ satisfying: \pause 
\begin{itemize}
\setbeamercolor{item projected}{bg=red!60!black,fg=white}
  \setbeamertemplate{enumerate items}[square]
    \item $A$ is quasi-isomorphic to the DG-algebra $C^{\bullet}(M)$.\pause 
    \item $A$ is connected, finite dimensional and satisfies Poincaré duality in dimension $m$ (i.e. there is an $A$-linear isomorphism $\theta \colon A \to A^{\lor}$ commuting with the differentials. 
\end{itemize}
\end{alertblock}
\pause 
\vspace{0.5cm}

$A$ will be referred to as a \textit{Poincaré duality model for $M$}.
\end{frame}

\begin{frame}{Idea of Proof}
Using the result of Lambrechts and Stanley, we first replace $C^{\bullet}(M)$ by its quasi-isomorphic DG-algebra $A$. \pause 

\vspace{0.5cm}

By a result in \cite{gerstenhaber} there is an isomorphism of Gerstenhaber algebras \pause 
$$HH^{\bullet}(A,A) \cong HH^{\bullet}(C^{\bullet}(M), C^{\bullet})) \, .$$
\pause 

\vspace{0.5cm}

Then, show that $HH_{\bullet}(A,A) \cong H^{\bullet}(\text{L}M)$ as an isomorphism of graded vector spaces. \pause 
\vspace{0.5cm}
\end{frame}
\begin{frame}{Idea of Proof (II)}
By the result of Menichi, the dual of Connes' operator $B^{\lor}$ on $HH^{\bullet+1}(A, A)^{\lor}$ gets transfered via the duality isomorphism 
$$\theta \colon HH_{\bullet}(A,A)^{\lor} \cong HH^{\bullet}(A,A^{\lor}) \xrightarrow{\simeq} HH^{\bullet}(A,A)$$
to a BV-algebra structure on $HH^{\bullet}(A,A)$ extending the Gerstenhaber algebra structure.
\vspace{0.5cm}

Finally, show that the isomorphism $HH_{\bullet}(A,A) \cong H^{\bullet}(\text{L}M)$ transfers Connes' operator $B$ to the operator $\Delta'$ induced by the action of $\mathbb{S}^1$ on $\text{L}M$ 
\end{frame}


\section[]{Hochschild complexes}
\begin{frame}{Hochschild (co)homology: Bar construction I}
Fix a DG-algebra $A$, a right $A$-module $P$ and a left $A$-module $N$. \pause 

\vspace{0.5cm}

Denote by $\text{Bar}(P,A,N)$ the following (two-sided) complex, \pause 
$$\text{Bar}_k(P,A,N)^l = (P \otimes T^k(A) \otimes N)^l \, .$$ 
\vspace{0.5cm}
Here, $k$ is the \textit{lower degree}. Given an element $p[a_1\mid a_2 \mid \cdots \mid a_k] n$ in $\text{Bar}_k(P,A,N)$, we will say it has \textit{upper degree} 
$$l = |p| + |n| + \sum_{i=1}^k |sa_i| \, .$$
\end{frame}

\begin{frame}{Hochschild (co)homology: Bar construction II}
We equip $\text{Bar}(P, A, N)$ with the differential $b_{\text{Bar}} = \partial_{DG} + \partial_{\text{Bar}}$: \pause 
\begin{align*}
\partial_{DG}(p[a_1 \mid \cdots \mid a_k] n) &= d(p)[a_1 \mid \cdots \mid a_k]n\\
& \quad - \sum_{i=1}^k (-1)^{\text{sign}_i} p[a_1 \mid \cdots \mid d(a_i)\mid \cdots \mid a_k] n \\
& \quad + (-1)^{\text{sign}_{k+1}}p[a_1 \mid \cdots \mid a_k]d(n) \, .
\end{align*}
\begin{align*}
\partial_{\text{Bar}}(p[a_1 \mid \cdots \mid a_k] n) & = (-1)^{|p|}p a_1[a_2 \mid \cdots \mid a_k] n \\
& \quad + \sum_{i=2}^k (-1)^{\text{sign}_i}p[a_1 \mid \cdots \mid |a_{i-1}a_i \mid \cdots \mid a_k] n \\
& \quad - (-1)^{\text{sign}_k} p[a_1 \mid \cdots \mid a_{k-1}] a_k n\, .
\end{align*} \pause 
Here, $\text{sign}_i$ denotes $|p|+ \sum_{j<i} |sa_j|$. 
\end{frame}


\begin{frame}{Hochschild (co)homology}
Let $A^e = A \otimes A^{\text{op}}$ and $P$ a DG right $A^e$-module. Define \pause 
$$C_{\bullet}(P,A) = P \otimes_{A^e} \text{Bar}(A,A,A) \, .$$\pause 
\vspace{0.5cm}
The complex $C_{\bullet}(P,A)$ is called the \textit{Hochschild chain complex of $A$ with coefficients in $P$}. Its homology is called the \textit{Hochschild homology of $A$ with coefficients in $P$}. \pause We denote it by $HH_{\bullet}(A,P)$.

\vspace{0.5cm}

Similarly, define for a left DG $A^e$-module $N$\pause 
$$C^{\bullet}(A, N) = \Hom_{A^e}(\text{Bar}(A,A,A), N) \, .$$
$C^{\bullet}(A,N)$ is the \textit{Hochschild cochain complex of $A$ with coefficients in $N$}. Its cohomology is called the \textit{Hochschild cohomology of $A$ with coefficients in $N$}. \pause We denote it by $HH^{\bullet}(N,A)$.
\end{frame}


\section[]{Gerstenhaber and BV-algebras}

\begin{frame}{Gerstenhaber algebras}
Recall that a \textit{Gerstenhaber algebra structure} on a commutative graded algebra $H = \{H_i\}_{i \in \mathbb{Z}}$ is given by a (Gerstenhaber) bracket 
$$[-,-] \colon H_i \otimes H_j \to H_{i+j +1} \, , \; x \otimes y \mapsto [x, y] \;\; .$$\pause
The bracket $[-,-]$ satisfies, for all $h, h', h'' \in H$: \pause 
\begin{itemize}
\setbeamercolor{item projected}{bg=red!60!black,fg=white}
  \setbeamertemplate{enumerate items}[square]
  \item $[h,h'] = (-1)^{(|h|-1)(|h'|-1)}[h', h]$, \pause 
  \item $[h, [h', h'']] = [[h,h'], h''] + (-1)^{(|a|-1)(|a'|-1)}[h', [h,h'']]$. 
\end{itemize}\pause 

\vspace{0.5cm}
Hochschild cohomology becomes a Gerstenhaber algebra via: \pause 
\begin{itemize}
\setbeamercolor{item projected}{bg=red!60!black,fg=white}
  \setbeamertemplate{enumerate items}[square]
  \item Cup product \pause (this gives graded commutativity),\pause 
  \item Gerstenhaber bracket: \pause circle product of Hochschild cochains.
\end{itemize}
\end{frame}


\begin{frame}{BV-algebras}
\pause Recall that a \textit{Batalin-Vilkovisky algebra} (BV-algebra) is a commutative graded algebra $H$ together with a linear map of degree $-1$ (BV-operator) \pause 
$$\Delta \colon H^k \to H^{k-1}$$
such that:\pause 
\begin{itemize}
    \setbeamercolor{item projected}{bg=red!60!black,fg=white}
  \setbeamertemplate{enumerate items}[square]
  \item $\Delta^2 = 0$, \pause
  \item $H$ becomes a Gerstenhaber algebra with bracket 
  $$[h,h'] = (-1)^{|h|}(\Delta(hh') - (-1)^{|h|}\Delta(h)h' - h\Delta(h') + h\Delta(1)h' \, .$$
\end{itemize}
\end{frame}

\begin{frame}{Connes' $B$ operator I}
The Hochschild complex carries a natural cyclic action induced by the cyclic bar construction. \pause 
\vspace{0.5cm}

Connes' $B$ operator is a degree $-1$ homogeneous map on Hocshchild chains that takes this action into account. \pause 

\vspace{0.5cm}

For a a general element $a_0 \otimes [a_1 \mid \cdots \mid a_n]$, one has \pause 
$$B(a_0\otimes [a_1 \mid \cdots \mid a_n]) = \sum_{i=0}^n (-1)^{\text{sign'}_i}1 \otimes [a_i \mid \cdots \mid a_n \mid a_0 \mid \cdots \mid a_{i-1}], ,$$
\pause where 
$$\text{sign'}_i = (|sa_0| + |sa_1| + \cdots + |sa_{i-1}|)(|sa_i| + \cdots + |sa_n|) \, .$$
\end{frame}

\section[]{BV-algebra structure on Hochschild cohomology}
\begin{frame}{Connes' $B$ operator II}
The Hochschild complex $C_{\bullet}(A)$ together with the operators $b$ and $B$ form a \textit{mixed complex} \pause ($B^2 = 0, B \circ b + b \circ B = 0$). \pause 
\vspace{0.5cm}

Its homology is known as the \textit{cyclic homology of $A$}, and it is denoted by $HC_{\bullet}(A)$. 
\end{frame}


\begin{frame}{BV-Structure in Hochschild cohomology}
Recall that $HH^{\bullet}(A)$ is a Gerstenhaber algebra where: \pause 
\begin{itemize}
\setbeamercolor{item projected}{bg=red!60!black,fg=white}
  \setbeamertemplate{enumerate items}[square]
    \item Graded commutativity comes from the cup product, \pause 
    \item Gerstenhaber algebra comes from circle product. 
\end{itemize}
\pause 
\vspace{0.5cm}

By duality, Connes' operator $B$ induces a map in cohomology
$$\overline{B} \colon HH^{n+1}(A) \to HH^n(A)\, .$$
\pause 
Here, $HH^{\bullet}(A) = HH^{\bullet}(A, A^{\lor})$.  \pause 
$$\overline{B}([f])([a_1 \otimes \cdots \otimes a_n])(a_0) = \sum_{i=0}^n \pm f(a_i \otimes \cdots \otimes a_n \otimes a_0 \otimes \cdots \otimes a_{i-1})(1) \, .$$\pause 
\begin{alertblock}{BV-algebra structure on Hochschild cohomology}
Hochschild cohomology with the dual of Connes' operator, $\overline{B}$, is a BV-algebra. 
\end{alertblock}
\end{frame}

\begin{frame}{Remarks}
The BV-algebra structure constructed above works more generally in the \textit{ungraded case}. \pause 
\vspace{0.5cm}

However, for our purposes, as we want to compare $HH^{\bullet}(C^{\bullet}, C^{\bullet})$ with the homology of the loop space, we quote the following result by L. Menichi (\cite{menichi})
\begin{alertblock}{Theorem}
The dual of Connes' operator $B$ together with the cup product induce a BV-algebra structure on Hochschild cohomology $H^{\bullet}(C^{\bullet}, C^{\bullet})$.
\end{alertblock}
\end{frame}

\section[]{BV-algebra structure on Loop Homology}
\begin{frame}{Chas-Sullivan Product}
Recall from previous talks, we have a loop product \pause
$$-\bullet - \colon LM \times LM \to LM\, .$$
\pause 
using the notion of \textit{transversal geometric chains}. \pause
\vspace{0.5cm}

Moreover, the circle action \pause 
$$\rho \colon \mathbb{S}^1 \times \text{L}M \to \text{L}M \; \; \rho(s, \alpha)(t) = \alpha(s+t)$$\pause
defines an operator $\Delta$ of degree $1$ commuting with the differential. \pause 
\vspace{0.5cm}
We have seen the following: \pause 
\begin{alertblock}{BV-structure on $\mathbb{H}_{\bullet}(\text{LM})$}
The loop product $\bullet$ together with the operator $\Delta$ define a BV-algebra structure on the loop homology $\mathbb{H}^{\bullet}(\text{L}M)$. 
\end{alertblock}
\end{frame}


\begin{frame}[fragile]{Chas-Sullivan Product (Dual)}
\pause We are interested in the \textit{dual} of the loop product $\bullet$. \pause Consider the diagram \pause 
\begin{equation*}
    \begin{tikzcd}
    \text{L}M^{\times 2} \arrow[d, "{(p_0, p_0}"] & \text{L}M \times_M \text{L}M \arrow[l, swap, "i"] \arrow[d, "p_0"] \arrow[r, "\text{Comp}"] & \text{L}M \arrow[d, "p_0"]\\
    M^{\times 2} & M \arrow[l, "\text{diag}"] \arrow[r, equal] & M
    \end{tikzcd}
\end{equation*}
\pause 
\vspace{0.5cm}
By Poincaré duality, given $f \colon E \to M$, we can construct the \textit{Gysin homomorphism} via the commutative diagram\pause 
\begin{equation*}
    \begin{tikzcd}
    H^k(E) \arrow[r, "f_*"] \arrow[d, "\text{P.D.}"] & H^{k - (e-m)}(M) \arrow[d, "\text{P.D.}"] \\
    H_{e-k}(E) \arrow[r, "f_*"] & H_{e-k}(M) 
    \end{tikzcd}
\end{equation*}
\pause 
where $e = \dim(E), m = \dim(M)$. 
\end{frame}

\begin{frame}[fragile]{Chas-Sullivan Product (Dual) II}
Taking $\text{diag} \colon M \to M \times M$ and $i \colon \text{L}M\times_M \text{L}M$, we obtain the Gysin maps\pause 
$$\text{diag}^! \colon H^k(M) \to H^{k+m}(M^{\times 2})\, ,\;\; i^{!} \colon H^k(\text{L}M \times_M \text{L}M) \to H^{k+m}(\text{L}M^{\times 2}) \, .$$

\vspace{0.5cm}
We obtain the following commutative diagram\pause 
\begin{equation*}
    \begin{tikzcd}
    H^{k+m}(\text{L}M^{\times 2}) & H^k(\text{L}M \times_M \text{L}M) \arrow[l, swap, "i^!"] & H^k(\text{L}M) \arrow[l, swap, "H^k(\text{Comp})"]\\
    H^{k+m}(M^{\times 2}\arrow[u, "H^*(p_0)^{\times 2}"] & H^k(M)\arrow[u, "{H^*(p_0)}"] \arrow[l, swap, "\text{diag}^!"]\arrow[r, equal] & H^k(M) \arrow[u, "H^*(p_0)"]
    \end{tikzcd}
\end{equation*}
\pause 
Define the \textit{dual of the loop product} as the composition 

$$i^! \circ H^{\bullet}(\text{Comp}) \colon H^*(\text{L}M) \to H^{\bullet+m}(\text{L}M^{\times 2}) \, .$$
\end{frame}


\begin{frame}{Remarks}
The construction of the dual loop product is not quite like that. \pause 

\vspace{0.5cm}

We need the notion of \textit{Thom spaces} and the \textit{Thom-Pontryagin construction}. \pause More concretely, the square formed by the maps $i$ and $\text{diag}$ is a pullback square which allows for a Thom-Pontryagin map 
$$\text{L}M \times \text{L}M \to (\text{L}M \times_M \text{L}M)^{TM}\, .$$ \pause 

\vspace{0.5cm}

Then, one argues in a similar fashion to define the dual of the loop product as before. 
\end{frame}

\section[]{Proof of Main Theorem}

\begin{frame}[fragile]{$C_{\bullet}(A,A)$ is a model of $\text{L}M$}
\pause
\begin{alertblock}{Proposition}
We have an isomorphism of graded vector spaces $HH_{\bullet}(A,A) \cong H^{\bullet}(\text{L}M)$. \pause The composite 
\begin{equation*}
    \begin{tikzcd}
    C_{\bullet}(A,A) \arrow[r] \arrow[d, equal] & C_{\bullet}(A,A) \otimes_A C_{\bullet}(A,A) \\
    A \otimes T(A) \arrow[r, "\ide \otimes \phi"] & A \otimes T(A) \otimes T(A) \arrow[u, "\simeq"]
    \end{tikzcd}
\end{equation*}
is a model of the composition of free loops. 
\end{alertblock}
\end{frame}

\begin{frame}[fragile]{Sketch of Proof}
Consider the diagram \pause
\begin{equation}
\label{comp}
    \begin{tikzcd}
	& {\text{L}M \times_M \text{L}M} &&& {M^I \times_M M^I} \\
	{\text{L}M} & {} & {} & {M^I} & {} \\
	& M && {} & {M^{\times3}} \\
	M &&& {M^{\times 2}}
	\arrow["{\text{Comp}}"', from=1-2, to=2-1]
	\arrow["{\text{Comp}^I}", from=1-5, to=2-4]
	\arrow["j", from=1-2, to=1-5]
	\arrow["j", from=2-1, to=2-4]
	\arrow["{(\text{id} \times \Delta) \circ \Delta}", from=3-2, to=3-5]
	\arrow["{p_0}"', from=2-1, to=4-1]
	\arrow["{(p_0,p_0,p_0)}"{pos=0.7}, shift left=2, from=1-5, to=3-5]
	\arrow["{\text{pr}}"', from=3-5, to=4-4]
	\arrow["\Delta"', from=4-1, to=4-4]
	\arrow[Rightarrow, no head, from=3-2, to=4-1]
	\arrow[from=2-4, to=4-4]
	\arrow["{p_0}"'{pos=0.6}, dashed, from=1-2, to=3-2]
    \end{tikzcd}
\end{equation}
\pause where $\text{Comp}^I$ denotes composition of paths, $\Delta$ the diagonal and $p_0$ denotes evaluation at $p_0 \in M$. \pause  
\end{frame}

\begin{frame}[fragile]{Sketch of Proof (II)}
Let $(A, d)$ be a commutative DG-algebra quasi-isomorphic to $C^{\bullet}(M)$. Consider 
\begin{equation}
\label{comp_right}
    \begin{tikzcd}
    \text{Bar}(A,A,A) \arrow[r, "\Psi"] & \text{Bar}(A,A,A) \otimes_A \text{Bar}(A,A,A) \\
    A^{\otimes 2} \arrow[r, "\psi"]\arrow[u] & A^{\otimes 3} \arrow[u]
    \end{tikzcd}
\end{equation}
\pause where 
$$\Psi(a \otimes [a_1 \mid \cdots \mid a_k] a') = \sum_{i=0}^k a \otimes [a_1 \mid \cdots \mid a_i] \otimes 1 \otimes [a_{i+1} \mid \cdots \mid a_k] \otimes a' \, ,$$
and \pause 
$$\psi(a \otimes a') = a \otimes 1 \otimes a' \, .$$
\pause Diagram \eqref{comp_right} is a cochain model of the right hand square in \eqref{comp}.
\end{frame}

\begin{frame}[fragile]{Sketch of Proof (III)}
    Tensoring by $A$ diagram \eqref{comp_right}, we obtain \pause 
    \begin{equation}
    \label{comp_left}
        \begin{tikzcd}
        A \otimes_{A^{\otimes 2}} \text{Bar}(A,A,A) \arrow[r, "\ide \otimes \Psi"]& A \otimes_{A^{\otimes 3}} (\text{Bar}(A,A,A) \otimes_A \text{Bar}(A,A,A)) \\
        A \otimes_{A^{\otimes 2}} A^{\otimes 2} \arrow[u] \arrow[r, "\ide \otimes \psi"]& A \otimes_{A^{\otimes 3}} A^{\otimes 3} \arrow[u]
        \end{tikzcd}
    \end{equation}
\pause Then, one realises that \eqref{comp_left} is actually a cochain model of the left hand square in \eqref{comp}. But we also have 
\begin{equation*}
    \begin{tikzcd}
    A \otimes_{A^{\otimes 2}} \text{Bar}(A,A,A) \arrow[r, "\ide \otimes \Psi"] & A \otimes_{A^{\otimes 3}} \text{Bar}(A,A,A) \otimes_A \text{Bar}(A,A,A) \\
    A \otimes T(A) \arrow[u, "\simeq"] \arrow[r, "\ide \otimes \phi"] & A \otimes T(A) \otimes T(A) \arrow[u, "\simeq"]
    \end{tikzcd}
\end{equation*}
\pause from which the Proposition follows. 
\end{frame}


\begin{frame}[fragile]
Consider the following diagram \pause 
\begin{equation}
\label{diag_1}
\begin{tikzcd}
A^{\lor} \arrow[r, "\mu^{\lor}"] &(A \otimes A)^{\lor} = A^{\lor} \otimes A^{\lor} \\
A  \arrow[u, "\theta"] \arrow[r, dotted, "\mu_A"] & A \otimes A \arrow[u, "\theta \otimes \theta"]
\end{tikzcd}
\end{equation}
\pause Here, $\mu \colon A \otimes A \to A$ denotes multiplication of $A$, $\theta$ is the dual isomorphism, and $\mu_A$ is defined by the commutative diagram. \pause 

\vspace{0.25cm}
We have the following
\begin{alertblock}{Lemma}
The map $\mu_A$ represents the Gysin map $\text{diag}^!$ and the map of degree $m$ \pause 
$$C_{\bullet}(A,A) \otimes_A C_{\bullet}(A,A) \xrightarrow{\simeq} A \otimes_{A^{\otimes 2}} C_{\bullet}(A,A)^{\otimes 2} \xrightarrow{\mu_A \otimes \ide} C_{\bullet}(A,A)^{\otimes 2}$$
commutes with the differential and induces $i^!$ in homology.
\end{alertblock}
\end{frame}


\begin{frame}[fragile]
Define a map $\Phi$ via the diagram 
\begin{equation}
\label{diag_2}
    \begin{tikzcd}
    A \otimes T(A) \arrow[d, dotted, "\Phi"] \arrow[r, "\ide \otimes \phi"] & A \otimes T(A) \otimes T(A) \cong A \otimes_{A^{\otimes 2}} (A \otimes T(A))^{\otimes 2} \arrow[d, "\mu_A \otimes \ide"] \\
    (A \otimes T(A))^{\otimes 2} & A^{\otimes 2} \otimes_{A^{\otimes 2}}(A \otimes T(A))^{\otimes 2} \arrow[l, "\simeq"]
    \end{tikzcd}
\end{equation}

\pause 
\vspace{0.5cm}
We have the following \pause 
\vspace{0.5cm}
\begin{alertblock}{Theorem}
The isomorphism $HH_{\bullet + m}(A,A)^{\lor} \cong HH^{\bullet + m}(A, A^{\lor}) \cong HH^{\bullet}(A,A)$ transfers $\Phi_*$ to the Gerstenhaber product on $HH^{\bullet}(A,A)$. 
\end{alertblock}
\end{frame}

\begin{frame}{References}
   \nocite{*}
   \bibliographystyle{plain}
   \bibliography{bibliography}
 \end{frame}
\end{document}
